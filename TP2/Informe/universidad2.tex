\subsection{UNISA}

%En este experimento decidimos realizar el traceroute hacia la universidad de UNISA, ubicada en Sudáfrica. Para la realización del mismo decidimos establecer un $TTL\_MAX$ de 30 saltos, es decir que cortabamos la ejecución si no alcanzabamos el destino en, a lo sumo, 30 saltos. Por otro lado, por cada iteración de los $TTL$ emitimos una ráfaga de 30 paquetes.

Comencemos por describir la ruta generada(Ver mapa que todavía no está). Se han realizado 27 saltos hasta alcanzar el destino solicitado, de los cuales en el $85 \% $ de los mismos aproximadamente, hemos obtenido respuestas del tipo $TIME\_EXCEEDED$, determinando, así que el largo de nuestra ruta es de 23 saltos, y que del resto de los valores del $TTL$ no hemos obtenido respuesta alguna. Como también se puede ver en el (mapa que no está), y el cuadro \ref{tabla-unisa} , durante el trayecto se realizaron 4 saltos intercontinentales, mientras que el método de Cimbala nos ha detectado en un principio todo número positivo como outlier y al sacar los 0s los que se observan en el cuadro antes mencionado.

\begin{table}[!htbp]
\centering
\caption{PONEME UN NOMBRE INTERESANTE UNISA}
\label{tabla-unisa}
\begin{tabular}{|c|c|c|c|}
\hline
\textbf{TTL} & \textbf{IP}     & \textbf{COUNTRY} & \textbf{OUTLIERS-CIMBALA} \\ \hline

  1	&	10.0.2.2			&	Undefined        &   	-\\ \hline
  6	&	200.89.161.129	&	Argentina        &   	outlier\\ \hline
  7	&	200.89.165.5		&	Argentina        &   	-\\ \hline
  8	&	200.89.165.250	&	Argentina        &   	-\\ \hline
  9	&	195.22.220.102	&	Italy            &   	-\\ \hline
 10	&	89.221.41.187	&	United States    &   	outlier\\ \hline
 11	&	89.221.41.187	&	United States    &   	-\\ \hline
 12	&	154.54.9.17		&	United States    &   	outlier\\ \hline
 13	&	154.54.80.41		&	United States    &   	-\\ \hline
 14	&	154.54.24.193	&	United States    &   	-\\ \hline
 15	&	154.54.7.157		&	United States    &   	outlier\\ \hline
 16	&	154.54.40.105	&	United States    &   	-\\ \hline
 17	&	154.54.30.186	&	United Kingdom   &   	outlier\\ \hline
 18	&	154.54.58.174	&	United Kingdom   &   	-\\ \hline
 19	&	154.54.56.242	&	United Kingdom   &   	-\\ \hline
 20	&	149.14.80.210	&	United Kingdom   &   	outlier\\ \hline
 21	&	196.32.209.174	&	South Africa     &   	outlier\\ \hline
 22	&	155.232.6.65		&	South Africa     &   	outlier\\ \hline
 23	&	155.232.6.37		&	South Africa     &   	-\\ \hline
 24	&	155.232.6.33		&	South Africa     &   	-\\ \hline
 25	&	155.232.6.142	&	South Africa     &   	-\\ \hline
 26	&	155.232.6.145	&	South Africa     &   	outlier\\ \hline
 27	&	155.232.6.138	&	South Africa     &   	-\\ \hline


\end{tabular}
\end{table}

En este contexto, nos disponemos a realizar un análisis más profundo de nuestros resultados. Podemos observar entonces, que de los cuatro saltos intercontinentales, Cimbala identifica correctamente todos, salvo el de Argentina a Italia, cuestión que podemos atribuirle a que, como muestra la figura \ref{fig:1}, el rtt de ese salto es 0. Por otro lado, 
 aunque identifica un par de saltos más, produciendo falsos positivos. Como se puede observar en la figura \ref{fig:1} los rtts correspondientes correspondientes a los saltos continentales tienen un valor claramente más elevado que el resto, con la excepción del primer salto, que tiene un rtt de 20 ms aproximadamente, numero no mayor al ttl 26 que no es un salto continental. Podríamos inferir que esta anomalía es la causante de que Cimbala identifique al ttl 26 como intercontinental cuando no lo es, pero esto

Al realizar el experimento, se logró llegar al destino en el ttl 27. De estos, 4 no respondieron. Los ttls que Cimbala reconoció como intercontinentales fueron 6, 10, 12, 15, 17, 20, 21, 22 y 26.

\begin{figure}[!htbp]
  \centering
    \includegraphics[scale=0.6]{imagenes/unisa-graficos/traceroute-unisa.jpg}
  \caption{UNISA- RTT hops}
  \label{fig:1}
\end{figure}

En la figura \ref{fig:1} se puede observar como el ttl 10, 17 y 21 tienen un rtt claramente distinguido del resto.

\begin{figure}[!htbp]
  \centering
    \includegraphics[scale=0.6]{imagenes/unisa-graficos/traceroute-unisa-standarized.jpg}
  \caption{UNISA- RTT hops standarized}
  \label{fig:2}
\end{figure}

En la figura \ref{fig:2} se puede observar una situacion similar a la de la figura \ref{fig:1}.

