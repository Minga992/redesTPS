\documentclass[a4paper]{article}
\usepackage[spanish]{babel}
\usepackage[utf8]{inputenc}
\usepackage{fancyhdr}
\usepackage{charter}   % tipografía
\usepackage{graphicx}
\usepackage{makeidx}

\usepackage{float}
\usepackage{amsmath, amsthm, amssymb}
\usepackage{amsfonts}
\usepackage{sectsty}
\usepackage{wrapfig}
\usepackage{listings} % necesario para el resaltado de sintaxis
\usepackage{caption}
\usepackage{placeins}
\usepackage{longtable}

\usepackage{hyperref} % agrega hipervínculos en cada entrada del índice
\hypersetup{          % (en el pdf)
    colorlinks=true,
    linktoc=all,
    citecolor=black,
    filecolor=black,
    linkcolor=black,
    urlcolor=black
}

\input{codesnippet}
\input{page.layout}
\usepackage{underscore}
\usepackage{caratula}
\usepackage{url}
\usepackage{color}
\usepackage{clrscode3e} % necesario para el pseudocodigo (estilo Cormen)




\begin{document}
%
%\lstset{
%  language=C++,                    % (cambiar al lenguaje correspondiente)
%  backgroundcolor=\color{white},   % choose the background color
%  basicstyle=\footnotesize,        % size of fonts used for the code
%  breaklines=true,                 % automatic line breaking only at whitespace
%  captionpos=b,                    % sets the caption-position to bottom
%  commentstyle=\color{red},    % comment style
%  escapeinside={\%*}{*)},          % if you want to add LaTeX within your code
%  keywordstyle=\color{blue},       % keyword style
%  stringstyle=\color{blue},     % string literal style
%}

\thispagestyle{empty}
\materia{Teoría de las Comunicaciones}
\submateria{Segundo Cuatrimestre 2016}
\titulo{TP1: Wiretapping}
%\subtitulo{Planos de Corte (INSERTE MEJORAS)}
\integrante{Cravero, Marcos}{495/15}{marcoscravero2175@gmail.com} % por cada integrante (apellido, nombre) (n° libreta) (e-mail)
\integrante{Mignanelli, Alejandro Rubén}{609/11}{minga_titere@hotmail.com} 
\integrante{Suárez, Federico}{610/11}{elgeniofederico@gmail.com} 

\maketitle
\newpage

\thispagestyle{empty}
\vfill
%\begin{abstract}
%    \vspace{0.5cm}
%	
%
%\end{abstract}

\thispagestyle{empty}
\vspace{1.5cm}
\tableofcontents
\newpage

%\normalsize
 
\newpage

\section{Introducción}
En el presente trabajo se utilizarán técnicas provistas por la Teoría de la Información para distinguir diversos aspectos de la red de manera analítica. Para esto se realizarán capturas de paquetes en 3 redes distintas modelando dichos paquetes con diferentes fuentes de información, S y S1, que se detallarán a continuación.
Sean $p_1..p_i$ los paquetes que se capturan en un enlace. Podemos conocer los destinos a los que están apuntados los paquetes que encapsulan la información con el campo $dst$ del frame de capa de enlace. Se pueden modelar los paquetes capturados como una fuente de información binaria de memoria nula S, definiendo el conjunto de símbolos que emite como {sBROADCAST, sUNICAST}. Si se captura un paquete $p$ en un intervalo de tiempo [$t_i$,$t_f$], se dice que S emite sBROADCAST si $p.dst$ == ff : ff : ff : ff : ff : ff, y si no emite sUNICAST. Esta fuente distingue entre mensajes Broadcast y Unicast que aparecen en la red en ese intervalo. 
Con el objetivo de distinguir los nodos (hosts) de la red se utilizará una fuente de información de memoria nula S1 que estará basada sólo en las direcciones IP de los paquetes ARP. En principio se tienen 3 opciones: tener en cuenta sólo las IPs destino, sólo las IPs origen o la combinación de ambas. Dado que se quieren distinguir los nodos de la red, se necesitaría que los símbolos de la fuente sean tales nodos, o más bien sus IPs. Con esto dicho, la tercera opción queda descartada ya que implicaría que los símbolos de la fuente sean combinaciones de pares de IPs.

\newpage

\section{Experimentación}
Para este proyecto se han realizado 3 capturas en distintos ámbitos: la Facultad de Ciencias Exactas de la UBA (la red de laboratorios del DC), un Starbucks y un Cyber. Para estas, se proponen 3 experimentos, que serán explicados más adelante.

\subsection{Redes Capturadas}

 A continuación se presentarán las condiciones de cada una de las capturas:

\begin{itemize}
	\item Facultad de Ciencias Exactas de la UBA, red de laboratorios del DC(desde ahora, La facultad):
	\begin{itemize}
		\item Horario: 16.50 hs aproximadamente.
		\item Cantidad de maquinas en la red: al menos 4.
		\item Duración de la captura: 20 minutos.
	\end{itemize}

	\item Starbucks:
	\begin{itemize}
		\item Ubicación: Callao y Viamonte.
		\item Horario: 15 hs.
		\item Cantidad de maquinas: al menos 3.
		\item Duración de la captura: 15 minutos aproximadamente.
	\end{itemize}		
	
	\item Cyber:
	\begin{itemize}
		\item Ubicación: Cabildo 2700.
		\item Cantidad de maquinas: 31 maquinas, 9 encendidas y conectadas a una red ethernet, 3 ocupadas.
		\item Duración de la captura: 15 minutos aproximadamente.
	\end{itemize}
	
\end{itemize}

\subsection{Experimentos propuestos}

A continuación se detallan los experimentos propuestos:

\begin{itemize}

	\item Utilizando la fuente $S_1$, se mostrará la cantidad de información de cada símbolo comparandola con la entropía de la fuente para darnos una idea de cuan informativo es cada símbolo, y así poder encontrar alguno que sea distinguido.
	\item Dados los paquetes ARP de la captura, se mostrará de la manera más clara e informativa posible la red de mensajes ARP subyacentes
	\item Utilizando la fuente binaria $S$, se mostrará la cantidad de infomación de cada símbolo(2) comparándolos con la entropía de la fuente y la entropía máxima, para poder analizar los mensajes broadcast de la red. 

\end{itemize}

\newpage

\section{Resultados}

A continuación se muestran los resultados obtenidos del experimento sobre cada universidad:

\subsection{Universidad1}

INSERTE RESULTADOS

\newpage

\subsection{Universidad2}

INSERTE RESULTADOS

\newpage

\subsection{Auckland}

Comencemos por describir la ruta generada(Ver mapa que todavía no está). Se han realizado 22 saltos hasta alcanzar el destino solicitado, de los cuales en el $72 \% $ de los mismos aproximadamente, hemos obtenido respuestas del tipo $TIME\_EXCEEDED$, determinando, así que el largo de nuestra ruta es de 16 saltos, y que del resto de los valores del $TTL$ no hemos obtenido respuesta alguna. Como también se puede ver en el (mapa que no está), durante el trayecto se realizaron 3 saltos intercontinentales, en los ttls 6, 10 Y 19. Por otro lado, el método de Cimbala nos ha detectado todo número positivo como outlier en primera instancia y al sacar los 0s nos devolvió como outliers los ttls 6, 10, 15, 16, 19, 22.


Al realizar el experimento, se logró llegar al destino en el ttl 22. De estos, 6 no respondieron. Los ttls que Cimbala reconoció como intercontinentales fueron 6, 10, 15, 16, 19, 22.

\begin{figure}[!htbp]
  \centering
    \includegraphics[scale=0.6]{imagenes/auckland-graficos/traceroute-auckland.jpg}
  \caption{auckland- RTT hops}
  \label{fig:7}
\end{figure}

En la figura \ref{fig:7} se puede observar como el ttl 10 y 19 tienen un rtt claramente distinguido del resto.

\begin{figure}[!htbp]
  \centering
    \includegraphics[scale=0.6]{imagenes/auckland-graficos/traceroute-auckland-standarized.jpg}
  \caption{auckland- RTT hops standarized}
  \label{fig:8}
\end{figure}

En la figura \ref{fig:8} se puede observar una situacion similar a la de la figura \ref{fig:7}.




\newpage

\subsection{Peking}

Comencemos por describir la ruta generada(Ver mapa que todavía no está). Se han realizado 29 saltos hasta alcanzar el destino solicitado, de los cuales, aproximadamente en el $86 \% $ de los mismos, hemos obtenido respuestas del tipo $TIME\_EXCEEDED$, determinando así, que el largo de nuestra ruta es de 25 saltos, y que del resto de los valores del $TTL$ no hemos obtenido respuesta alguna. Como también se puede ver en el (mapa que no está), durante el trayecto se realizaron 3 saltos intercontinentales, correspondientes a los hops 9, 10 Y 20.

% MAPAS ACA

Analicemos ahora los outliers detectados por Cimbala y contrastemos los datos con la realidad.

\begin{table}[!htbp]
\centering
\caption{Traceroute Peking}
\label{traceroute-peking}
\begin{tabular}{|c|c|c|c|}
\hline
\textbf{TTL} & \textbf{IP}   			     & \textbf{COUNTRY} & \textbf{OUTLIERS} \\ \hline
1            & 10.0.2.2                       & Undefined        &                   \\ \hline
6            & 200.89.161.149                 & Argentina        & {[}outlier{]}     \\ \hline
7            & 200.89.165.130                 & Argentina        &                   \\ \hline
8            & 200.89.165.222                 & Argentina        &                   \\ \hline
9            & 185.70.203.32                  & Italy            &                   \\ \hline
10           & 89.221.41.181                  & United States    & {[}outlier{]}     \\ \hline
11           & 89.221.41.181                  & United States    &                   \\ \hline
12           & 154.54.9.17                    & United States    & {[}outlier{]}     \\ \hline
13           & 154.54.80.41                   & United States    &                   \\ \hline
14           & 66.28.4.237                    & United States    & {[}outlier{]}     \\ \hline
15           & 154.54.29.222                  & United States    & {[}outlier{]}     \\ \hline
16           & 154.54.42.77                   & United States    &                   \\ \hline
17           & 154.54.45.162                  & United States    &                   \\ \hline
18           & 154.54.45.2                    & United States    &                   \\ \hline
19           & 38.88.196.186                  & United States    &                   \\ \hline
20           & 101.4.117.169                  & China            & {[}outlier{]}     \\ \hline
21           & 101.4.117.97                   & China            & {[}outlier{]}     \\ \hline
22           & 101.4.117.50                   & China            &                   \\ \hline
23           & 101.4.115.69                   & China            &                   \\ \hline
24           & 101.4.118.94                   & China            &                   \\ \hline
25           & 101.4.112.90                   & China            &                   \\ \hline
26           & 101.4.117.81                   & China            &                   \\ \hline
27           & 202.112.41.178                 & China            &                   \\ \hline
28           & 202.112.41.182                 & China            &                   \\ \hline
29           & 162.105.252.133                & China            &                   \\ \hline
\end{tabular}
\end{table}


Como se puede observar en la tabla \ref{traceroute-peking}, el método propuesto por Cimbala ha determinado que existen 7 saltos cuyos tiempos son outliers. Como hemos mencionado previamente, tan solo se han producido 3 saltos entre continentes, dos de los cuales se han detectado correctamente, mientras que el noveno salto, entre Argentina e Italia, no ha sido identificado. Por lo tanto, podemos afirmar que el método ha presentado 5 casos de falsos positivos, es decir, casos que señaló como saltos intercontnentales y no lo eran, y un caso de falso negativo, donde no ha determinado que un salto sea entre continentes cuando si lo fue.

En cuanto al caso del falso negativo, es decir el salto entre Argentina e Italia, esto se debe a que el tiempo para ese hop, como se puede ver en la figura \ref{fig:10}, fue 0, con lo cual, el mismo pudo haber sido efectivamente 0 o más bien un valor negativo que hemos reemplazado con un cero. Este valor, que ya lo hemos visto en el experimento de otra universidad, puede deberse a algunos de los factores que hemos planteado al principio del presente documento en donde indicamos algunas hipótesis del por qué de tiempos negativos. \\
Más allá de eso, cabe mencionar que, si el tiempo que se corresponde con el salto es nulo, y que para la aplicación del método de Cimbala hemos decidido no contemplar los ceros, es claro que jamás lo iba a poder detectar como un outlier. Simplemente en esta ocasión el problema radicó en haber obtenido un tiempo negativo para dicho hop.

En resumen, si contamos todo outlier como salto intercontinental, obtenemos en esta muestra:

\begin{itemize}
	\item $20 \% $ de falsos positivos.
	\item $4 \%$ de falsos negativos.
	\item $76 \%$ de resultados acertados.
\end{itemize}

Realizando un análisis parecido al de Oxford, en la figura \ref{fig:11} , podemos apreciar que la mayoría de los puntos se sitúan un poco por debajoo del 0.4 y cualquier otro punto que no respete esta línea, es considerado outlier. Nuevamente observemos, que si nos quedamos solo con los puntos que se encuentran por encima de este valor, obtenemos exactamente los saltos intercontinentales, con excepción del de Argentina a Italia. 

\begin{figure}[!htbp]
  \centering
    \includegraphics[scale=0.5]{imagenes/peking-graficos/traceroute-peking.jpg}
  \caption{peking- RTT hops}
  \label{fig:10}
\end{figure}


\begin{figure}[!htbp]
  \centering
    \includegraphics[scale=0.5]{imagenes/peking-graficos/traceroute-peking-standarized.jpg}
  \caption{peking- RTT hops standarized}
  \label{fig:11}
\end{figure}

Resumiendo, si consideramos solo los valores que en la figura \ref{fig:11} están por encima del punto de acumulaciónPONGANME UN BUEN NOMBRE, pasamos a obtener un $96\%$ de aciertos, siendo el $4\%$ restante un caso muy borde.


\newpage

\newpage

\section{Conlusiones}

ESCRIBA AQUI SUS CONCLUSIONES

\newpage

\end{document}

