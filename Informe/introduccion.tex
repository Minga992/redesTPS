\section{Introducción}
En el presente trabajo se utilizarán técnicas provistas por la Teoría de la Información para distinguir diversos aspectos de la red de manera analítica. Para esto se realizarán capturas de paquetes en 3 redes distintas modelando dichos paquetes con diferentes fuentes de información, S y S1, que se detallarán a continuación.
Sean $p_1..p_i$ los paquetes que se capturan en un enlace. Podemos conocer los destinos a los que están apuntados los paquetes que encapsulan la información con el campo $dst$ del frame de capa de enlace. Se pueden modelar los paquetes capturados como una fuente de información binaria de memoria nula S, definiendo el conjunto de símbolos que emite como {sBROADCAST, sUNICAST}. Si se captura un paquete $p$ en un intervalo de tiempo [$t_i$,$t_f$], se dice que S emite sBROADCAST si $p.dst$ == ff : ff : ff : ff : ff : ff, y si no emite sUNICAST. Esta fuente distingue entre mensajes Broadcast y Unicast que aparecen en la red en ese intervalo. 
Con el objetivo de distinguir los nodos (hosts) de la red se utilizará una fuente de información de memoria nula S1 que estará basada sólo en las direcciones IP de los paquetes ARP. En principio se tienen 3 opciones: tener en cuenta sólo las IPs destino, sólo las IPs origen o la combinación de ambas. Dado que se quieren distinguir los nodos de la red, se necesitaría que los símbolos de la fuente sean tales nodos, o más bien sus IPs. Con esto dicho, la tercera opción queda descartada ya que implicaría que los símbolos de la fuente sean combinaciones de pares de IPs.