\section{Experimentación}
Para este proyecto se han realizado 3 capturas en distintos ámbitos: la Facultad de Ciencias Exactas de la UBA (la red de laboratorios del DC), un Starbucks y un Cyber. Para estas, se proponen 3 experimentos, que serán explicados más adelante.

\subsection{Redes Capturadas}

 A continuación se presentarán las condiciones de cada una de las capturas:

\begin{itemize}
	\item Facultad de Ciencias Exactas de la UBA, red de laboratorios del DC(desde ahora, La facultad):
	\begin{itemize}
		\item Horario: 16.50 hs aproximadamente.
		\item Cantidad de maquinas en la red: al menos 4.
		\item Duración de la captura: 20 minutos.
	\end{itemize}

	\item Starbucks:
	\begin{itemize}
		\item Ubicación: Callao y Viamonte.
		\item Horario: 15 hs.
		\item Cantidad de maquinas: al menos 3.
		\item Duración de la captura: 15 minutos aproximadamente.
	\end{itemize}		
	
	\item Cyber:
	\begin{itemize}
		\item Ubicación: Cabildo 2700.
		\item Cantidad de maquinas: 31 maquinas, 9 encendidas y conectadas a una red ethernet, 3 ocupadas.
		\item Duración de la captura: 15 minutos aproximadamente.
	\end{itemize}
	
\end{itemize}

\subsection{Experimentos propuestos}

A continuación se detallan los experimentos propuestos:

\begin{itemize}

	\item Utilizando la fuente $S_1$, se mostrará la cantidad de información de cada símbolo comparandola con la entropía de la fuente para darnos una idea de cuan informativo es cada símbolo, y así poder encontrar alguno que sea distinguido.
	\item Dados los paquetes ARP de la captura, se mostrará de la manera más clara e informativa posible la red de mensajes ARP subyacentes con el objetivo de identificar algún nodo que destaque frente a los demás en cuanto a la cantidad de conexiones que recibe y a su vez se desprenden de este.
	\item Utilizando la fuente binaria $S$, se mostrará la cantidad de infomación de cada símbolo(2) comparándolos con la entropía de la fuente y la entropía máxima, para poder analizar los mensajes broadcast de la red. 

\end{itemize}